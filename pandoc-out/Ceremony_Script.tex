\large
\hypertarget{section}{%
\section{Anja \& David Wedding --- Ceremony Script}\label{section}}


\hypertarget{setting}{%
\subsection{Setting}\label{setting}}

\begin{itemize}
\tightlist
\item
  On the dais: Mark, parents, aunts, others?
\item
  DJ in place
\item
  PA system functional
\end{itemize}

\hypertarget{processional}{%
\subsection{1. Processional:}\label{processional}}

\begin{enumerate}
\def\labelenumi{\arabic{enumi}.}
\tightlist
\item
  Cue music
\item
  Children of the bubbles
\item
  David \& Anja walk in together, holding hands
\item
  Anja \& David take their place of honor on the dais, facing each
  other.
\item
  Music fade-out
\end{enumerate}

\hypertarget{remarks-from-mark}{%
\subsection{2. Remarks from Mark}\label{remarks-from-mark}}

\emph{Welcome! Thank you so much for being here.}

\emph{Turns out, as you might guess, it's kinda a big deal to be asked
to officiate a wedding ceremony.}

\emph{After David asked me to do this, I really had no idea where to
start. I began bringing it up with various people in my life to try to
get a sense of what is required here. And that resulted in some better,
and some worse, advice on what makes a good wedding. But maybe the best
advice I got was from Aunt Linda who passed on advice she received about
her own wedding, which was that you need to remember that this is a kind
of a show: this is the Anja and David show --- And it needs to be
planned and executed like a show, with a narrative arc, rising action,
and of course a climax.}

\emph{This idea appealed to me because if we think of the wedding as a
show, it puts the focus on the audience. A show is FOR the audience
after all, in this case, that is you. The whole point of this wedding
venture, I think, is to do it in front of people who matter to David and
Anja. That means YOU have all the power here. Most especially Grandma.
YOU are the witnesses who affirm the marriage. And take the significance
of the ceremony with you. And there's really no power vested in me as an
individual even though I'm standing up here; the power of this thing is
really vested in YOU, the audience.}

\emph{Conveniently, this really lets me off the hook for what at first
seemed like a tremendously grave responsibility.}

\emph{Not entirely, of course. If it's a show we're putting on, then I'm
still stuck with the role of narrator, or maybe story-teller. And I sort
of struggled with that --- here's the problem: the first rule of telling
a good story or putting on a good show is to NOT give away the ending.
You have to pique the curiosity of the audience, give them a bunch of
clues, and only slowly reveal the significance of the clues. That's
where the pleasure of hearing a story comes from: the hints of mystery,
the audience's desire to find out what is going to happen.}

\emph{Well, you might begin to see the problem here: at a wedding, the
ending is already given away. We all KNOW what's going to happen!
There's no plot twists here, no jump scares, no surprises. Hopefully.}

\emph{And then I thought: maybe that's what a good wedding needs: more
mystery, more tension, more clues --- something like that part of the
movie Jaws where the camera is half in the water, and sneaking slowly up
on a swimmer, and the music is really slow (duh nun, duh nuh) because
the tension is only just starting to build\ldots{} who could ever turn
away at a moment like that -- you just GOTTA find out what's going to
happen. Wouldn't it be awesome to be at a wedding that had that same
power to give you the sense that something dramatic is going to happen?}

\emph{So that made me wonder: how can I make this wedding be more like
Jaws?}

\emph{But then, Jaws is a movie that is almost 50 years old now. I have
no memory of the first time I saw it. But I still love that movie. David
and I have probably seen it a hundred times each, but I'll watch it
again with any one of you any time you like. And no matter how good the
tension-building effects --- the music, the camera angles, the slow
reveal of the shark --- no matter how good the experience is for someone
who is seeing it for the first time, there's gotta be something more to
that movie. Something that makes it special enough that you can watch it
50 times and still enjoy it.}

\emph{This is something we do in our family: we watch the movies we like
over and over again. So like, even though we know every line in ``Star
Trek II: Wrath of Khan'' every camera angle, every reaction shot ---
we'll still want to watch it at least once a year. And I realized: well,
that's just a different kind of indicator of a good story --- even when
you know a story so well that there's no mystery or curiosity left,
there's a pleasure in the anticipation of the familiar, don't you think?
You enjoy it BECAUSE you know what's coming rather than because you
don't know what's coming. You look for your favorite parts, and you feel
satisfaction to find out it's just as good as the previous 50 times you
saw it. Maybe, in some way, better, because you're affirming that your
favorite movie wasn't just some passing fancy --- you weren't just a
victim of the latest special effects or topical jokes --- your favorite
movie is one that has stood the test of time. The movie is timeless. And
is there anything more reassuring in life than the quality of
timelessness?}

\emph{And I thought: well if it works for Jaws, maybe that's how a
wedding works too: we all get the satisfaction of a familiar story told
over again. We know what the good parts are and that they are inevitably
coming to us. We're all sitting here in anticipation\ldots{}
anticipation that absolutely nothing unexpected will happen. And I don't
think there are going to be any surprises today --- though if there are,
then rather than blaming me, I just encourage you to enjoy the
unexpected plot twist!}

\emph{But maybe we can push this idea even further --- though forgive me
this if it's going too far. Maybe this idea applies to more than just
the wedding ceremony; maybe a lot of the best marriages are founded on
that kind of anticipation of the familiar. After all, if you are still
finding anticipation, satisfaction, and reassurance in the same person
after decades, what more could you ask from a relationship? So I hope
you find that in your marriage Anja and David.}

\emph{But I can't leave it at that. The story-teller part of me also hopes
that your marriage is more like your experience when you first read Lord
of the Rings: a super-long, exciting, page-turner of epic proportions.
Like maybe this is just the beginning of the story, and you are just
reading the first couple of pages here at this wedding. You're just
getting to know these characters David and Anja. And, really, they are
just getting to know each other. Then there is a whole lifetime ahead of
them to get the rest of the story arc down. I hope it will be one full
of twists and turns and surprises. And seriously: not all good ones. If
people have to overcome adversity, it makes for a better story, right?
Though I still hope you never have to deal with a giant shark. But
putting aside some of the more experimental fiction in the world, mostly
we want to see the characters we love successfully overcome the
obstacles and navigate the plot twists and come out on top.}

\emph{So that's what I wish for you David and Anja\ldots{} both a really
good story with lots of excitement and surprises. And also a story that
you keep coming back to and finding comfort in. That, I think, is kinda
everything a marriage ever could be. I hope it's yours.}

\hypertarget{a-few-short-readings}{%
\subsection{3. A few short readings}\label{a-few-short-readings}}

\begin{enumerate}
\def\labelenumi{\arabic{enumi}.}
\tightlist
\item
  Monika
\item
  Becci
\item
  Ted LOTR
\end{enumerate}

\hypertarget{i-dos}{%
\subsection{4. I dos}\label{i-dos}}

\begin{enumerate}
\def\labelenumi{\arabic{enumi}.}
\tightlist
\item
  Mark Says: \emph{And now, let's get to some rising action. Are you
  both ready?}
\item
  \emph{Do you, Anja, take David to be your husband?}

  \begin{itemize}
  \tightlist
  \item
    Anja: \emph{I do}
  \end{itemize}
\item
  \emph{Do you David, take Anja to be your wife?}

  \begin{itemize}
  \tightlist
  \item
    David: \emph{I do}
  \end{itemize}
\end{enumerate}

\hypertarget{exchange-of-rings-vows}{%
\subsection{5. Exchange of rings + vows}\label{exchange-of-rings-vows}}

\begin{enumerate}
\def\labelenumi{\arabic{enumi}.}
\tightlist
\item
  Mark says: \emph{As I'm sure you are all expecting, we'll move on to
  the vows as the plot continues to thicken. Do we have the rings?}
\item
  Anja places the ring on David's finger and says her vows/nice words.

  \begin{itemize}
  \tightlist
  \item
    And then again in German
  \end{itemize}
\item
  David places the ring on Anja's finger and says his vows/nice words.

  \begin{itemize}
  \tightlist
  \item
    And then again in German
  \end{itemize}
\end{enumerate}

\hypertarget{fishermans-knot}{%
\subsection{6. Fisherman's knot}\label{fishermans-knot}}

\begin{enumerate}
\def\labelenumi{\arabic{enumi}.}
\tightlist
\item
  Mark says: \emph{David and Anja have decided that a fitting symbol of
  their marriage will be an attempt to literally tie the knot. The knot
  they will be attempting is known as a `fisherman's knot' which has the
  characteristic that the more you pull on the ends, the tighter it
  gets. Get it? It's a metaphor for marriage\ldots{}}
\item
  David and Anja hopefully don't make a fool of themselves tying a big
  knot.
\end{enumerate}

\hypertarget{the-smooch}{%
\subsection{7. The Smooch}\label{the-smooch}}

\begin{itemize}
\tightlist
\item
  David and Anja continue to hold the knot in their hands.
\end{itemize}

\begin{enumerate}
\def\labelenumi{\arabic{enumi}.}
\tightlist
\item
  Mark says: \emph{And now the climax --- what you've all been
  anticipating: the smooches! Anja and David, you may smooch}
\item
  David and Anja smooch.
\item
  Applause
\end{enumerate}

\hypertarget{the-pronouncement-and-presentation}{%
\subsection{The pronouncement and
presentation}\label{the-pronouncement-and-presentation}}

\begin{enumerate}
\def\labelenumi{\arabic{enumi}.}
\tightlist
\item
  Music starts: love keeps lifting me higher
\item
  Mark says: \emph{I get to say now that Anja and David, you are husband
  and wife!}
\item
  They kiss again or whatever.
\item
  Mark says: \emph{And the denouement: It is my great pleasure to
  present to you the married couple: David and Anja!}
\end{enumerate}

\hypertarget{recessional}{%
\subsection{Recessional}\label{recessional}}

\begin{enumerate}
\def\labelenumi{\arabic{enumi}.}
\tightlist
\item
  music continues to play, louder.
\item
  David and Anja walk out together (what does everyone else do?)
\item
  2nd, chiller, song ready for long run-out.
\item
  Pops takes the mic to guide people to the champagne toast.
\end{enumerate}
